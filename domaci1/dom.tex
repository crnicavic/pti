\documentclass[12pt]{article}
\usepackage{geometry}
\geometry{
	a4paper,
	lmargin=2cm,
	rmargin=2cm,
	tmargin=2cm,
	bmargin=2cm
}
\usepackage{float}
\usepackage{indentfirst}
\usepackage{amsmath}
\usepackage{caption}
\usepackage{graphicx}
\usepackage{listings}
\usepackage{color}
\usepackage[thinc]{esdiff}
\usepackage[T1]{fontenc}
\renewcommand{\contentsname}{Sadržaj}
\title{Primenjena teorija igara domaći 1 - Karnoovo takmičenje N kompanija}
\author{Ognjen Čavić E2 161/2024}
\date{Novembar 2024}
\begin{document}
\maketitle
\section{Nomenklatura}
\noindent
$q_i$ - Količina proizvoda koja je proizvedena od strane $i$-te kompanije u
takmičenju \\[0.2cm]
$c_i$ - Ukupna cena proizvodnje odgovarajuće kompanije \\[0.2cm]
$d_i$ - Cena ulaska na tržište \\[0.2cm]
$\overline{c_i}$ - Koeficijent proizvodnje kompanije tj. trošak pravljenja
jednog proizvoda \\[0.2cm]
$p(q_1,..., q_N)$ - Cena pojedinačnog proizvoda koji zavisi od ukupne količine
proizvoda \\[0.2cm]
$a$ - Faktor osetljivosti cene proizvoda \\[0.2cm]
$b$ - Maksimalna cena proizvoda \\[0.2cm]
$u_i(q_1,..., q_2)$ - Profit tj. dobit svake od kompanija
\section{Model u slučaju kada se N kompanija takmiči}
\par Jedina stvar koja se razlikuje u odnosu na model sa dve kompanije je to
što se sabiraju količine proizvoda više kompanija.
Kako bi se stvari malo pojednostavile pretpostavlja se da su cene proizvodnje
pojedinačnih proizvoda i ulaska na tržište iste za svaku kompaniju tj:
\begin{align*}
	c_1 &= c_2 = ... = c_N \\
	d_1 &= d_2 = ... = d_N 
\end{align*}
Stoga su konačne cene proizvodnje za svaku od kompanija sledeće:
\begin{align*}
	c_1(q_1) &= \overline{c} q_1 + d \\
	c_2(q_2) &= \overline{c} q_2 + d \\
	% ja sam jebeni genije
		&\hspace{0.7cm}\vdots \\
	c_N(q_N) &= \overline{c} q_N + d 
\end{align*}
Odakle se vidi se da u ovom slučaju cene proizvodnje zavise samo od količine
robe.
Pojedinačna cena svakog proizvoda je opisana na sledeći način:
\begin{equation}
	p(q_1, q_2, ..., q_N) = b - a \sum_{i=1}^N q_i
\end{equation}
Preostaje definisati funkciju koja opisuje profit:
\begin{align*}
	u_1(q_1, q_2, ..., q_N) &= q_1 * p(q_1, q_2, ..., q_N) - c_1(q_1) \\
	u_2(q_1, q_2, ..., q_N) &= q_2 * p(q_1, q_2, ..., q_N) - c_2(q_2) \\
		&\hspace{0.7cm}\vdots \\
	u_N(q_1, q_2, ..., q_N) &= q_N * p(q_1, q_2, ..., q_N) - c_N(q_N)
\end{align*}
\section{Analiza promene plasirane robe, pojedinačne cene i dobiti sa porastom
broja kompanija}
\par Cilj je maksimizovati profit, stoga je potrebno uraditi prvi izvod profita
odgovarajuće kompanije po količini plasirane robe jer je to jedina stvar kojom
ta kompanija može da upravlja.
\begin{align*}
	\frac{\partial u_1}{\partial q_1} &= b - a q_1 - a \sum_{i=1}^N q_i
	- \overline{c} = 0\\ 
	\frac{\partial u_2}{\partial q_2} &= b - a q_2 - a \sum_{i=1}^N q_i
	- \overline{c} = 0\\ 
		&\hspace{1.8cm}\vdots \\
	\frac{\partial u_N}{\partial q_N} &= b - a q_N - a \sum_{i=1}^N q_i
	- \overline{c} = 0 \\ 
\end{align*}
Sabiranjem ovih jednačina dobija se sledeće:
\begin{gather*} 	
	n b - a \sum_{i=1}^N q_i - a N \sum_{i=1}^N q_i
	- N \overline{c} = 0 \\
	n b - a (N+1) \sum_{i=1}^N q_i - N \overline{c} = 0 \\
\end{gather*}
Odakle sledi optimalna količina proizvoda koja dovodi do maksimalne dobiti:
\begin{equation}
	q^* = \sum_{i=1}^N q_i = \frac{N b - N \overline{c}}{a (N+1)}
\end{equation}
gde je neophodan uslov $b > \overline{c}$ jer broj proizvoda ne sme biti
negativan.
Zamenom u izraz za pojedinačnu cenu plasirane robe, dobija se i optimalna cena
pojedinačnog proizvoda:
\begin{equation}
	p^* (q_1, q_2, ..., q_N) = \frac{b + N \overline{c}}{N+1}
\end{equation}
\section{Analiza slučaja kada broj kompanija neograničeno raste}
\par Kada broj kompanija neograničeno raste potrebno je ispitati ka kojim
vrednostima konvergiraju broj proizvoda i pojedinačna cena:
\begin{align*}
	\lim_{N \rightarrow \infty} q^*
	&= \lim_{N \rightarrow \infty} \frac{N b - N \overline{c}}{a (N+1)}
	= \frac{b - \overline{c}}{a} \\
	\lim_{N \rightarrow \infty} p^*
	&= \lim_{N \rightarrow \infty} \frac{b + N \overline{c}}{N+1}
	= \overline{c}
\end{align*}
što znači da porast broja kompanija preko nekog broja dovodi do toga da se
cena i količina proizvoda skoro i ne menjaju, a to podrazumeva da se isti
profit deli između više kompanija tj. da svaka kompanija ima manju dobit tj.
profit.
\section{Karnoov duopol sa nesimetričnim troškovima proizvodnje}
\par Dosadašnji rad se bavio slučajem kada postoji proizvoljan broj kompanija
N koje su simetrične tj. imaju iste troškove proizvodnje, 
\end{document}
