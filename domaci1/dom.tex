\documentclass[12pt]{article}
\usepackage{geometry}
\geometry{
	a4paper,
	lmargin=2cm,
	rmargin=2cm,
	tmargin=2cm,
	bmargin=2cm
}
\usepackage{float}
\usepackage{indentfirst}
\usepackage{amsmath}
\usepackage{caption}
\usepackage{graphicx}
\usepackage{cancel}
\usepackage{listings}
\usepackage{color}
\usepackage[thinc]{esdiff}
\usepackage[T1]{fontenc}
\renewcommand{\contentsname}{Sadržaj}
\title{Primenjena teorija igara domaći 1 - Karnoovo takmičenje N kompanija}
\author{aa}
\date{Novembar 2024}
\begin{document}
\maketitle
\section{Nomenklatura}
\noindent
$q_i$ - Količina proizvoda koja je proizvedena od strane $i$-te kompanije u
takmičenju \\[0.2cm]
$c_i$ - Ukupna cena proizvodnje odgovarajuće kompanije \\[0.2cm]
$d_i$ - Cena ulaska na tržište \\[0.2cm]
$\overline{c_i}$ - Koeficijent proizvodnje kompanije tj. trošak pravljenja
jednog proizvoda \\[0.2cm]
$p(q_1,..., q_N)$ - Cena pojedinačnog proizvoda koji zavisi od ukupne količine
proizvoda \\[0.2cm]
$a$ - Faktor osetljivosti cene proizvoda \\[0.2cm]
$b$ - Maksimalna cena proizvoda \\[0.2cm]
$u_i(q_1,..., q_2)$ - Profit tj. dobit svake od kompanija
\section{Model u slučaju kada se N kompanija takmiči}
\par Jedina stvar koja se razlikuje u odnosu na model sa dve kompanije je to
što se sabiraju količine proizvoda više kompanija.
Kako bi se stvari malo pojednostavile pretpostavlja se da su cene proizvodnje
pojedinačnih proizvoda i ulaska na tržište iste za svaku kompaniju tj:
\begin{align*}
	c_1 &= c_2 = ... = c_N \\
	d_1 &= d_2 = ... = d_N 
\end{align*}
Stoga su konačne cene proizvodnje za svaku od kompanija sledeće:
\begin{align*}
	c_1(q_1) &= \overline{c} q_1 + d \\
	c_2(q_2) &= \overline{c} q_2 + d \\
	% ja sam jebeni genije
		&\hspace{0.7cm}\vdots \\
	c_N(q_N) &= \overline{c} q_N + d 
\end{align*}
Odakle se vidi se da u ovom slučaju cene proizvodnje zavise samo od količine
robe.
Pojedinačna cena svakog proizvoda je opisana na sledeći način:
\begin{equation}
	p(q_1, q_2, ..., q_N) = b - a \sum_{i=1}^N q_i
\end{equation}
Preostaje definisati funkciju koja opisuje profit:
\begin{align*}
	u_1(q_1, q_2, ..., q_N) &= q_1 * p(q_1, q_2, ..., q_N) - c_1(q_1) \\
	u_2(q_1, q_2, ..., q_N) &= q_2 * p(q_1, q_2, ..., q_N) - c_2(q_2) \\
		&\hspace{0.7cm}\vdots \\
	u_N(q_1, q_2, ..., q_N) &= q_N * p(q_1, q_2, ..., q_N) - c_N(q_N)
\end{align*}
\section{Analiza promene plasirane robe, pojedinačne cene i dobiti sa porastom
broja kompanija}
\par Cilj je maksimizovati profit, stoga je potrebno uraditi prvi izvod profita
odgovarajuće kompanije po količini plasirane robe jer je to jedina stvar kojom
ta kompanija može da upravlja.
\begin{align*}
	\frac{\partial u_1}{\partial q_1} &= b - a q_1 - a \sum_{i=1}^N q_i
	- \overline{c} = 0\\ 
	\frac{\partial u_2}{\partial q_2} &= b - a q_2 - a \sum_{i=1}^N q_i
	- \overline{c} = 0\\ 
		&\hspace{1.8cm}\vdots \\
	\frac{\partial u_N}{\partial q_N} &= b - a q_N - a \sum_{i=1}^N q_i
	- \overline{c} = 0 \\ 
\end{align*}
Sabiranjem ovih jednačina dobija se sledeće:
\begin{gather*} 	
	n b - a \sum_{i=1}^N q_i - a N \sum_{i=1}^N q_i
	- N \overline{c} = 0 \\
	n b - a (N+1) \sum_{i=1}^N q_i - N \overline{c} = 0 \\
\end{gather*}
Odakle sledi optimalna količina proizvoda koja dovodi do maksimalne dobiti:
\begin{equation}
	q^* = \sum_{i=1}^N q_i = \frac{N b - N \overline{c}}{a (N+1)}
\end{equation}
gde je neophodan uslov $b > \overline{c}$ jer broj proizvoda ne sme biti
negativan.
Zamenom u izraz za pojedinačnu cenu plasirane robe, dobija se i optimalna cena
pojedinačnog proizvoda:
\begin{equation}
	p^* (q_1, q_2, ..., q_N) = \frac{b + N \overline{c}}{N+1}
\end{equation}
\section{Analiza slučaja kada broj kompanija neograničeno raste}
\par Kada broj kompanija neograničeno raste potrebno je ispitati ka kojim
vrednostima konvergiraju broj proizvoda i pojedinačna cena:
\begin{align*}
	\lim_{N \rightarrow \infty} q^*
	&= \lim_{N \rightarrow \infty} \frac{N b - N \overline{c}}{a (N+1)}
	= \frac{b - \overline{c}}{a} \\
	\lim_{N \rightarrow \infty} p^*
	&= \lim_{N \rightarrow \infty} \frac{b + N \overline{c}}{N+1}
	= \overline{c}
\end{align*}
što znači da porast broja kompanija preko nekog broja dovodi do toga da se
cena i količina proizvoda skoro i ne menjaju, a to podrazumeva da se isti
profit deli između više kompanija tj. da svaka kompanija ima manju dobit tj.
profit. 
%ovde bih mogao ubaciti nesto oko totalnog profita
\section{Karnoov duopol sa nesimetričnim troškovima proizvodnje}
\par Dosadašnji rad se bavio slučajem kada postoji proizvoljan broj kompanija
N koje su simetrične tj. imaju iste troškove proizvodnje, sada je potrebno
ispitati šta se dešava kada jedna kompanija ima manje troškove proizvodnje.
\begin{align*}
	c_1 (q_1) &= \overline{c_1} q_1 + d_1 \\
	c_2 (q_2) &= \overline{c_2} q_2 + d_2
\end{align*}
Cena pojedinačnih proizvoda se definiše na identičan način kao i do sada:
\begin{equation*}
	p(q_1, q_2) = b - a (q_1 + q_2)
\end{equation*}
funkcija profita se takođe ne menja:
\begin{align*}
	u_1(q_1, q_2) &= q_1 p - c_1 = q_1 b - a q_1^2 - a q_1 q_2 - \overline{c_1}
	- d_1\\
	u_2(q_1, q_2) &= q_2 p - c_2 = q_2 b - a q_2^2 - a q_1 q_2 - \overline{c_2}
	- d_2\\
\end{align*}
traži se prvi izvod po količini plasirane robe svake kompanije:
\begin{align*}
	\frac{\partial u_1}{\partial q_1} &= b - 2 a q_1 - a q_2 - \overline{c_1} = 0
	\\
	\frac{\partial u_2}{\partial q_2} &= b - a q_1 - 2 a q_2 - \overline{c_1} = 0
\end{align*}
sabiranjem ove dve jednačine se dobija izraz iz kog se određuje optimalna
količina proizvoda:
\begin{gather*}
	2 b - 3 a q_1 - 3 a q_2 - \overline{c_1} - \overline{c_2} = 0 \\
	q^* = q_1 + q_2 = \frac{2 b - \overline{c_1} - \overline{c_2}}{3 a}
\end{gather*}
odakle sledi optimalna cena pojedinačnog proizvoda:
\begin{equation*}
	p^*(q^*) = b
	- \cancel{a} \frac{2 b - \overline{c_1} - \overline{c_2}}{3 \cancel{a}}
	= \frac{b + \overline{c_1} + \overline{c_2}}{3}
\end{equation*}
odakle je moguće videti da koeficijenti proizvodnje utiču na optimalnu cenu
proizvoda. Primera radi, pretpostavlja se da firma 1 ima manje troškove
proizvodnje što to je opisano sledećim odnosom:
\begin{equation*}
	\overline{c_1} = a \overline{c_2} \Rightarrow a > 0 \land a < 1
\end{equation*}
i potom optimalna cena pojedinačnog proizvoda je jednaka:
\begin{equation*}
	p^* = \frac{b + (1+a) \overline{c_2}}{3}
\end{equation*}
i može se videti da je ona manja nego kad su koeficijenti proizvodnje jednaki.
Odatle sledi da kompanija sa jeftinijom proizvodnjom dobija prednost na tržištu
jer za manje novca može da proizvede više i ima veću zaradu po proizvodu.
% jel ovo dobar zakljucak?
\end{document}
